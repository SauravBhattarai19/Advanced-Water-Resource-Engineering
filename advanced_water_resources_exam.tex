\documentclass[11pt,a4paper]{article}
\usepackage[margin=2cm]{geometry}
\usepackage{amsmath}
\usepackage{amssymb}
\usepackage{array}
\usepackage{booktabs}
\usepackage{enumitem}
\usepackage{multicol}
\usepackage{graphicx}
\usepackage{multirow}

\title{\textbf{Advanced Water Resources Engineering\\
Midterm Examination}}
\author{Fall 2025}
\date{Duration: 20 minutes \quad Total Marks: 50}

\begin{document}

\maketitle

\vspace{-0.5cm}

\section*{Instructions}
\begin{itemize}[nosep]
    \item This examination consists of two sections with a total duration of \textbf{20 minutes}.
    \item \textbf{Section A}: 10 multiple choice questions, 3 marks each (30 marks total)
    \item \textbf{Section B}: 3 numerical problems
    \item \textbf{Total marks}: 50
    \item Formula table is provided on the last page
    \item Show all calculations clearly for numerical problems
    \item Use the answer sheet provided for multiple choice questions
\end{itemize}
\vspace{0.8cm}



Name and J\_Number: 
\vspace{0.1cm}

\hrule
\vspace{0.3cm}

\section*{Section A: Multiple Choice Questions (30 marks)}
\textbf{Choose the best answer for each question. Each question is worth 3 marks.}

\begin{enumerate}

\item In the Weibull plotting position method, if you have 25 years of data and an event is ranked 5th largest, what is its plotting position P?
\begin{enumerate}[label=\Alph*)]
    \item P = 5/25 = 0.20
    \item P = 5/26 = 0.19
    \item P = 5/24 = 0.21
    \item P = 4/25 = 0.16
\end{enumerate}

\item Which statement best distinguishes probability from frequency?
\begin{enumerate}[label=\Alph*)]
    \item Probability and frequency are the same thing
    \item Frequency is observed count; probability is long-term expectation
    \item Probability changes with more data; frequency stays constant
    \item Frequency is theoretical; probability is observed
\end{enumerate}

\item If the risk of failure is 0.35, what is the reliability?
\begin{enumerate}[label=\Alph*)]
    \item 0.35
    \item 0.65
    \item 1.35
    \item Cannot be determined
\end{enumerate}

\item What is the main difference between PDF and CDF?
\begin{enumerate}[label=\Alph*)]
    \item PDF shows probability density; CDF shows cumulative probability
    \item PDF is for continuous data; CDF is for discrete data
    \item PDF values can exceed 1; CDF values cannot exceed 1
    \item Both A and C are correct
\end{enumerate}

\item Given the following goodness-of-fit test results for annual maximum precipitation:
\begin{center}
\begin{tabular}{lc}
Distribution & KS test p-value \\
\hline
Normal & 0.12 \\
Log-Normal & 0.45 \\
Exponential & 0.02 \\
Gumbel & 0.38 \\
\end{tabular}
\end{center}
Which distribution provides the best fit ($\alpha$ = 0.05)?
\begin{enumerate}[label=\Alph*)]
    \item Normal
    \item Log-Normal
    \item Exponential
    \item Gumbel
\end{enumerate}

\item A GEV distribution is fitted to flood data and the shape parameter $\xi = +0.25$. This indicates:
\begin{enumerate}[label=\Alph*)]
    \item Light tail (Weibull type) - extremes are bounded
    \item Standard tail (Gumbel type) - normal extreme behavior
    \item Heavy tail (Fréchet type) - extreme events more likely
    \item The distribution fitting failed
\end{enumerate}

\item In IDF curves, what does the "I" represent?
\begin{enumerate}[label=\Alph*)]
    \item Total rainfall depth (mm)
    \item Rainfall intensity (mm/hr)
    \item Infrastructure design standard
    \item Infiltration rate (mm/hr)
\end{enumerate}

\item A "100-year flood" occurred twice in a 10-year period. Which statement is correct?
\begin{enumerate}[label=\Alph*)]
    \item It's no longer a 100-year flood; it's now a 5-year flood
    \item The 100-year designation is wrong and should be recalculated
    \item This is natural variability; the long-term probability remains 1\%
    \item This indicates climate change has altered the flood regime
\end{enumerate}

\item Why are longer return periods generally selected for critical infrastructure?  
\begin{enumerate}[label=\Alph*)]
    \item To minimize the cost of construction  
    \item To reduce the probability of failure and increase safety  
    \item Because rainfall data is always available for long durations  
    \item To simplify hydrologic analysis  
\end{enumerate}

\item If 30 mm of rain falls in 45 minutes, what is the rainfall intensity?
\begin{enumerate}[label=\Alph*)]
    \item 30 mm/hr
    \item 40 mm/hr
    \item 45 mm/hr
    \item 67 mm/hr
\end{enumerate}

\end{enumerate}

\newpage

\section*{Section B: Numerical Problems (20 marks)}

\begin{enumerate}

\item \textbf{IDF Curve Application (8 marks)}

From an IDF curve, the 100-year return period rainfall intensity for 30 minutes duration is 85 mm/hr. Calculate the peak discharge for a watershed with the following characteristics:
\begin{itemize}[nosep]
    \item Drainage area (A) = 3.5 hectares
    \item Runoff coefficient (C) = 0.75
    \item Use the rational method: $Q = \frac{1}{3600} \times C \times I \times A$ (where I is in mm/hr, A is in hectares)
\end{itemize}

Express your answer in L/s.

\vspace{8cm}

\item \textbf{Risk and Reliability Analysis (6 marks)}

A storm drainage system is designed for a 25-year return period storm with a design life of 40 years. Calculate:
\begin{enumerate}[label=\alph*)]
    \item The annual exceedance probability
    \item The lifetime risk
    \item The reliability
\end{enumerate}

Show whether this design meets the acceptable risk criterion of $\leq 30\%$ for commercial infrastructure.

\vspace{8cm}

\item \textbf{Return Period Calculation (6 marks)}

From 35 years of annual maximum flood data, the 3rd largest event has a magnitude of 450 m$^3$/s. Using the Weibull plotting position method:
\begin{enumerate}[label=\alph*)]
    \item Calculate the plotting position P
    \item Calculate the return period T
    \item State what this return period means in practical terms
\end{enumerate}

\vspace{7cm}

\end{enumerate}

\newpage

\section*{Formula Table}

\renewcommand{\arraystretch}{1.5}
\begin{tabular}{|l|l|l|}
\hline
\textbf{Category} & \textbf{Formula} & \textbf{Description} \\
\hline
\multirow{3}{*}{\textbf{Frequency Analysis}}
& $P = \frac{m}{n+1}$ & Weibull plotting position \\
& $T = \frac{1}{P}$ & Return period \\
& $P = \frac{1}{T}$ & Annual probability \\
\hline
\multirow{3}{*}{\textbf{Risk \& Reliability}}
& $R = 1-(1-P)^n$ & Lifetime risk \\
& $Rel = (1-P)^n$ & Reliability \\
& $R + Rel = 1$ & Risk-reliability relationship \\
\hline
\multirow{3}{*}{\textbf{Probability}}
& $0 \leq P \leq 1$ & Probability range \\
& $P(X \geq a) = 1 - F(a)$ & Exceedance probability \\
& $P(a \leq X \leq b) = F(b) - F(a)$ & Interval probability \\
\hline
\multirow{2}{*}{\textbf{IDF Curves}}
& $I = \frac{P}{t}$ & Intensity calculation \\
& $Q = \frac{1}{3600} \times C \times I \times A$ & Rational method (I in mm/hr, A in hectares) \\
\hline
\multirow{3}{*}{\textbf{GEV Distribution}}
& $\xi > 0$ & Heavy tail (Fréchet) \\
& $\xi = 0$ & Gumbel distribution \\
& $\xi < 0$ & Light tail (Weibull) \\
\hline
\end{tabular}

\vspace{0.5cm}

\textbf{Where:}
\begin{multicols}{2}
\begin{itemize}[nosep]
    \item $P$ = Probability
    \item $m$ = Rank of observation
    \item $n$ = Total observations
    \item $T$ = Return period (years)
    \item $R$ = Risk
    \item $Rel$ = Reliability
    \item $I$ = Rainfall intensity (mm/hr)
    \item $Q$ = Peak discharge
    \item $C$ = Runoff coefficient
    \item $A$ = Drainage area (hectares)
    \item $t$ = Duration
    \item $F(x)$ = Cumulative distribution function
    \item $\xi$ = GEV shape parameter
    \item $\mu$ = Mean
    \item $\sigma$ = Standard deviation
    \item $Z$ = Standard normal variable
\end{itemize}
\end{multicols}

\vspace{0.5cm}

\textbf{Unit Conversions:}
\begin{itemize}[nosep]
    \item 1 hectare = 10,000 m$^2$
    \item Intensity (mm/hr) = $\frac{\text{Depth (mm)}}{\text{Duration (hr)}}$
    \item To convert minutes to hours: divide by 60
    \item To convert m$^3$/s to L/s: multiply by 1000
\end{itemize}

\end{document}