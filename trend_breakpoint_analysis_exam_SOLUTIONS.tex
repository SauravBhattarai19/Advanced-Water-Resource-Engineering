\documentclass[11pt,letterpaper]{article}
\usepackage[margin=1in]{geometry}
\usepackage{amsmath,amssymb}
\usepackage{graphicx}
\usepackage{enumitem}
\usepackage{booktabs}
\usepackage{array}
\usepackage{fancyhdr}
\usepackage{lastpage}
\usepackage{tikz}
\usepackage{xcolor}
\usepackage{tcolorbox}

% Header and Footer
\pagestyle{fancy}
\fancyhf{}
\lhead{CIV 465: Advanced Water Resources Engineering}
\rhead{Exam Solutions}
\cfoot{Page \thepage\ of \pageref{LastPage}}

% Custom commands
\newcommand{\points}[1]{\textbf{[#1 points]}}
\newcommand{\answer}[1]{\textcolor{blue}{\textbf{#1}}}
\newcommand{\explanation}[1]{\begin{tcolorbox}[colback=green!5,colframe=green!40!black,title=Explanation]\small #1\end{tcolorbox}}

\begin{document}

% Title Section
\begin{center}
    \Large\textbf{Trend Detection and Breakpoint Analysis Exam}\\[0.2cm]
    \large\textcolor{red}{\textbf{COMPLETE SOLUTIONS WITH EXPLANATIONS}}\\[0.3cm]
    \large CIV 465: Advanced Water Resources Engineering\\
    Jackson State University\\[0.3cm]
\end{center}

\vspace{0.5cm}

%==============================================================================
\section*{Part A: Multiple Choice Questions \points{30}}
%==============================================================================

\begin{enumerate}[leftmargin=*]

\item A hydrologic time series dataset has a strongly skewed distribution with several extreme outliers. Which test is most appropriate for detecting monotonic trends?

\answer{Answer: (b) Mann-Kendall test}

\explanation{
The Mann-Kendall test is a \textbf{non-parametric} test that:
\begin{itemize}
    \item Uses \textbf{ranks} instead of actual values, making it robust to outliers
    \item Does NOT require normality assumption (works with skewed data)
    \item Recommended by WMO (World Meteorological Organization) for hydrologic data
\end{itemize}
Linear regression (option a) assumes normality and is heavily influenced by outliers. The t-test (option c) also assumes normality. Pearson correlation (option d) measures linear association and is sensitive to outliers.
}

\vspace{0.3cm}

\item In the Mann-Kendall test, if the S statistic equals +180 for a dataset with 25 observations, and the calculated p-value is 0.08, what should you conclude at $\alpha = 0.05$?

\answer{Answer: (b) Fail to reject $H_0$; no significant trend detected}

\explanation{
Decision Rule: Compare p-value to significance level $\alpha$
\begin{itemize}
    \item p-value = 0.08 $>$ $\alpha$ = 0.05 $\Rightarrow$ Fail to reject $H_0$
    \item This means the apparent upward pattern (S = +180) could reasonably occur by chance
    \item Note: It WOULD be significant at $\alpha$ = 0.10 (since 0.08 $<$ 0.10)
\end{itemize}
The positive S indicates an increasing pattern, but it's not statistically significant at the 0.05 level. We need stronger evidence (lower p-value) to claim a real trend exists.
}

\vspace{0.3cm}

\item Which statement correctly describes the difference between trends and change points in hydrologic data?

\answer{Answer: (b) Trends are gradual monotonic changes; change points are abrupt shifts}

\explanation{
\textbf{Key Distinction:}
\begin{itemize}
    \item \textbf{Trends:} Gradual, monotonic (consistently up or down) changes over time
        \begin{itemize}
            \item Detected by Mann-Kendall test
            \item Example: Slow increase in temperature due to climate change
        \end{itemize}
    \item \textbf{Change Points:} Sudden, abrupt shifts in mean level
        \begin{itemize}
            \item Detected by Pettitt test
            \item Example: Dam construction immediately changing downstream flow
        \end{itemize}
\end{itemize}
This distinction matters for engineering design: trends require projecting future values, while change points require splitting the dataset.
}

\vspace{0.3cm}

\item You calculate Sen's slope = +2.5 mm/year with 95\% CI = [+1.8, +3.2] for annual rainfall. The Mann-Kendall p-value is 0.02. What does this indicate?

\answer{Answer: (b) Significant increasing trend with high confidence in magnitude}

\explanation{
Three pieces of evidence work together:
\begin{enumerate}
    \item \textbf{Sen's slope = +2.5 mm/year:} Positive $\Rightarrow$ Increasing trend
    \item \textbf{p-value = 0.02 $<$ 0.05:} Statistically significant trend
    \item \textbf{95\% CI = [+1.8, +3.2]:}
        \begin{itemize}
            \item Narrow range $\Rightarrow$ High confidence in the magnitude
            \item Does NOT include zero $\Rightarrow$ Confirms real positive change
        \end{itemize}
\end{enumerate}
This is an ideal result: the trend is both statistically significant AND well-quantified with tight confidence bounds.
}

\vspace{0.3cm}

\item For a Pettitt test, what does the $U_{t,T}$ statistic represent?

\answer{Answer: (b) The sum of pairwise rank comparisons between segments before and after time $t$}

\explanation{
The $U_{t,T}$ statistic is calculated as:
$$U_{t,T} = \sum_{i=1}^{t} \sum_{j=t+1}^{T} \text{sgn}(X_i - X_j)$$

This is essentially a \textbf{Mann-Whitney statistic} that:
\begin{itemize}
    \item Compares ALL observations before time $t$ with ALL observations after time $t$
    \item Uses the sign function: +1 if $X_i > X_j$, -1 if $X_i < X_j$, 0 if equal
    \item Large positive $U$ = values before $t$ tend to be larger (downward shift)
    \item Large negative $U$ = values before $t$ tend to be smaller (upward shift)
\end{itemize}
It's rank-based (non-parametric), not based on means or variances.
}

\vspace{0.3cm}

\item A stream discharge dataset (40 years) shows: Mann-Kendall p = 0.03, Sen's slope = +2.1 m$^3$/s/year. The post-change period after a detected dam construction has only 3 years of data. What is the most appropriate engineering action?

\answer{Answer: (c) Document the change point but use regional pooling or conservative design factors}

\explanation{
This is a practical engineering judgment question:
\begin{itemize}
    \item The change point is \textbf{real} (p = 0.03 $<$ 0.05)
    \item But 3 years is \textbf{insufficient} for reliable frequency analysis (need at least 10-20 years)
\end{itemize}

\textbf{Best Practice:}
\begin{enumerate}
    \item Document the change point and its physical cause (dam)
    \item Use data from similar regulated watersheds (regional pooling)
    \item Apply conservative safety factors to account for uncertainty
    \item Plan to re-analyze as more post-change data accumulates
\end{enumerate}

Using only 3 years (option a) gives unreliable statistics. Ignoring the change (option b) uses outdated data. Waiting indefinitely (option d) is impractical.
}

\end{enumerate}

\newpage

%==============================================================================
\section*{Part B: Short Numerical Problems \points{45}}
%==============================================================================

\subsection*{Problem 1: Mann-Kendall S Statistic Calculation \points{15}}

\textbf{Given Data:}

\begin{center}
\begin{tabular}{|c|c|c|c|c|c|c|c|c|}
\hline
Year & 1 & 2 & 3 & 4 & 5 & 6 & 7 & 8 \\
\hline
Rainfall (mm) & 850 & 870 & 865 & 890 & 900 & 895 & 920 & 930 \\
\hline
\end{tabular}
\end{center}

S contributions: +7, +4, +5, +2, +1, +2, +1

\vspace{0.3cm}

\textbf{(a) Calculate the total Mann-Kendall S statistic.} \points{3}

\answer{Solution:}
$$S = 7 + 4 + 5 + 2 + 1 + 2 + 1 = \boxed{+22}$$

\explanation{
The S statistic is simply the sum of all pairwise comparison results. Since S = +22 (positive), this indicates an \textbf{increasing trend} in rainfall. The more positive S is, the stronger the upward trend.
}

\vspace{0.3cm}

\textbf{(b) Calculate the variance of S.} \points{4}

\answer{Solution:}
$$\text{Var}(S) = \frac{n(n-1)(2n+5)}{18}$$

With $n = 8$:
$$\text{Var}(S) = \frac{8 \times 7 \times (2 \times 8 + 5)}{18} = \frac{8 \times 7 \times 21}{18} = \frac{1176}{18} = \boxed{65.33}$$

\explanation{
This formula assumes no tied values. The variance tells us how much S would vary if there were no trend (under $H_0$). It's used to standardize S into a Z-score for determining statistical significance.
}

\vspace{0.3cm}

\textbf{(c) Calculate the standardized Z-score.} \points{4}

\answer{Solution:}

Since $S > 0$, we use:
$$Z = \frac{S - 1}{\sqrt{\text{Var}(S)}} = \frac{22 - 1}{\sqrt{65.33}} = \frac{21}{8.083} = \boxed{2.60}$$

\explanation{
The "-1" is a continuity correction that improves accuracy. The Z-score follows a standard normal distribution under $H_0$. A Z-score of 2.60 means the observed S is 2.60 standard deviations away from what we'd expect if there were no trend.
}

\vspace{0.3cm}

\textbf{(d) What is your conclusion about the trend?} \points{4}

\answer{Solution:}

Since $|Z| = 2.60 > 1.96$:
\begin{itemize}
    \item \textbf{Reject $H_0$} at $\alpha = 0.05$
    \item There is a \textbf{statistically significant increasing trend} in rainfall
    \item p-value $\approx$ 0.009 (for Z = 2.60)
\end{itemize}

\explanation{
\textbf{Critical Values:}
\begin{itemize}
    \item $|Z| > 1.645$: Significant at $\alpha = 0.10$
    \item $|Z| > 1.96$: Significant at $\alpha = 0.05$ (standard threshold)
    \item $|Z| > 2.576$: Significant at $\alpha = 0.01$
\end{itemize}
Our Z = 2.60 $>$ 2.576, so the trend is actually significant even at $\alpha = 0.01$ level. This is strong evidence of increasing rainfall over the 8-year period.
}

\newpage

\subsection*{Problem 2: Sen's Slope Estimator \points{15}}

\textbf{Given pairwise slopes:}

$Q_{12} = 8.00$, $Q_{13} = 5.00$, $Q_{14} = 6.00$, $Q_{15} = 6.25$, $Q_{23} = 2.00$

$Q_{24} = 5.00$, $Q_{25} = 5.67$, $Q_{34} = 8.00$, $Q_{35} = 7.50$, $Q_{45} = 7.00$

\vspace{0.3cm}

\textbf{(a) Sort all 10 slopes in ascending order.} \points{3}

\answer{Solution:}
$$\boxed{2.00, 5.00, 5.00, 5.67, 6.00, 6.25, 7.00, 7.50, 8.00, 8.00}$$

\explanation{
Sorting is necessary because Sen's slope uses the \textbf{median}, which requires ordered data. Notice we have two 5.00 values and two 8.00 values - this is expected when multiple pairs have similar slopes.
}

\vspace{0.3cm}

\textbf{(b) Calculate the Sen's slope estimator $\beta$.} \points{4}

\answer{Solution:}

With $n = 10$ slopes (even number), the median is the average of the 5th and 6th values:

$$\beta = \frac{Q_{(5)} + Q_{(6)}}{2} = \frac{6.00 + 6.25}{2} = \boxed{6.125 \text{ m}^3\text{/s/year}}$$

\explanation{
\textbf{Why median instead of mean?}
\begin{itemize}
    \item Median is robust to outliers (extreme slopes don't bias the result)
    \item Mean would be: $(2+5+5+5.67+6+6.25+7+7.5+8+8)/10 = 6.04$ (similar here)
    \item For skewed data with outliers, median is more representative
\end{itemize}
The median slope represents the "typical" rate of change in the data.
}

\vspace{0.3cm}

\textbf{(c) Interpret the meaning of this slope in engineering terms.} \points{4}

\answer{Solution:}

\begin{itemize}
    \item Peak discharge is \textbf{increasing at 6.125 m$^3$/s per year}
    \item Over 10 years: increase of $6.125 \times 10 = 61.25$ m$^3$/s
    \item This represents approximately $61.25/100 = 61\%$ increase from Year 1
\end{itemize}

\explanation{
\textbf{Engineering Implications:}
\begin{itemize}
    \item If this trend continues, infrastructure designed on historical data may become undersized
    \item Flood risk is increasing - design standards may need updating
    \item Possible causes: urbanization (more impervious surface), climate change, land use changes
    \item This is a substantial rate of change that warrants investigation
\end{itemize}
}

\vspace{0.3cm}

\textbf{(d) Projected discharge in Year 10.} \points{4}

\answer{Solution:}

Method 1 - From Year 5:
$$Q_{10} = Q_5 + \beta \times (10 - 5) = 125 + 6.125 \times 5 = 125 + 30.625 = \boxed{155.6 \text{ m}^3\text{/s}}$$

Method 2 - From Year 1:
$$Q_{10} = Q_1 + \beta \times (10 - 1) = 100 + 6.125 \times 9 = 100 + 55.125 = \boxed{155.1 \text{ m}^3\text{/s}}$$

\explanation{
Both methods give approximately the same answer (small difference due to rounding). Sen's slope allows us to:
\begin{itemize}
    \item Project future values assuming trend continues
    \item Estimate historical values at times between measurements
    \item Calculate total change over any time period
\end{itemize}
\textbf{Caution:} Extrapolating assumes the trend continues unchanged. For long-term projections (like 50+ years), consider uncertainty and whether the physical process driving the trend will persist.
}

\newpage

\subsection*{Problem 3: Pettitt Test for Change Point Detection \points{15}}

\textbf{Given $U_{t,T}$ values:}

\begin{center}
\begin{tabular}{|c|c|c||c|c|c|}
\hline
$t$ & $U_{t,T}$ & $|U_{t,T}|$ & $t$ & $U_{t,T}$ & $|U_{t,T}|$ \\
\hline
1 & -8 & 8 & 7 & -42 & 42 \\
2 & -18 & 18 & 8 & -36 & 36 \\
3 & -26 & 26 & 9 & -28 & 28 \\
4 & -32 & 32 & 10 & -18 & 18 \\
5 & -38 & 38 & 11 & -6 & 6 \\
6 & -44 & \textcolor{red}{\textbf{44}} & & & \\
\hline
\end{tabular}
\end{center}

\vspace{0.3cm}

\textbf{(a) Determine the test statistic $K_\tau$.} \points{3}

\answer{Solution:}
$$K_\tau = \max_{1 \le t < T} |U_{t,T}| = \max(8, 18, 26, 32, 38, 44, 42, 36, 28, 18, 6) = \boxed{44}$$

\explanation{
$K_\tau$ is the maximum absolute value of U across all potential split points. This represents the "strongest evidence" for a change point anywhere in the series. The larger $K_\tau$, the more evidence for a real change point.
}

\vspace{0.3cm}

\textbf{(b) Identify the change point location $\tau$.} \points{3}

\answer{Solution:}

The maximum $|U_{t,T}| = 44$ occurs at $t = 6$, therefore:
$$\boxed{\tau = 6}$$

This means the change point is \textbf{after the 6th observation} (i.e., data splits into first 6 vs. last 6 observations).

\explanation{
The change point location $\tau$ tells us WHERE in the time series the most significant shift occurs. In this case:
\begin{itemize}
    \item Observations 1-6: One regime (e.g., pre-change period)
    \item Observations 7-12: Another regime (e.g., post-change period)
\end{itemize}
If this were annual data from 1985-1996, $\tau = 6$ means the change occurred after 1990.
}

\vspace{0.3cm}

\textbf{(c) Calculate the p-value.} \points{5}

\answer{Solution:}

Given: $K_\tau = 44$, $T = 12$

$$p \approx 2 \exp\left(-\frac{6K_\tau^2}{T^3 + T^2}\right)$$

\textbf{Step 1: Calculate denominator}
$$T^3 + T^2 = 12^3 + 12^2 = 1728 + 144 = 1872$$

\textbf{Step 2: Calculate numerator}
$$6K_\tau^2 = 6 \times 44^2 = 6 \times 1936 = 11616$$

\textbf{Step 3: Calculate ratio}
$$\frac{11616}{1872} = 6.205$$

\textbf{Step 4: Calculate p-value}
$$p \approx 2 \times \exp(-6.205) = 2 \times 0.00202 = \boxed{0.00404 \approx 0.004}$$

\explanation{
This formula is an asymptotic approximation valid for $T \geq 20$ (though we use it here for demonstration). The exponential function decreases rapidly as $K_\tau$ increases, meaning larger test statistics give smaller p-values (stronger evidence for change point).

\textbf{Calculation tip:} $e^{-6.205} \approx 0.002$ (use calculator or $e^{-6} \approx 0.0025$)
}

\vspace{0.3cm}

\textbf{(d) Is there a significant change point? What does negative $U_\tau$ indicate?} \points{4}

\answer{Solution:}

\textbf{Significance:}
$$p = 0.004 < 0.05 \Rightarrow \text{\textbf{Yes, highly significant change point}}$$

\textbf{Direction of change:}

$U_\tau = -44$ (negative) means:
\begin{itemize}
    \item Values \textbf{before} $\tau$ tend to be \textbf{smaller} than values after
    \item This indicates an \textbf{upward shift} (increase) in the data
    \item Example: Mean flow increased after year 6
\end{itemize}

\explanation{
\textbf{Interpreting the sign of $U_\tau$:}
\begin{itemize}
    \item $U_\tau < 0$ (negative): \textbf{Upward shift} - values increase after change point
    \item $U_\tau > 0$ (positive): \textbf{Downward shift} - values decrease after change point
\end{itemize}

With p = 0.004 $<$ 0.01, this is actually significant even at the 1\% level, providing very strong evidence for a real change point.

\textbf{Engineering action:} Split the dataset at $\tau = 6$, use post-change data (observations 7-12) for design if it represents current conditions.
}

\newpage

%==============================================================================
\section*{Part C: Analytical Interpretation Questions \points{25}}
%==============================================================================

\subsection*{Question 1 \points{12}}

\textbf{Given Results:}
\begin{itemize}
    \item Mann-Kendall: S = +892, p = 0.0003
    \item Sen's Slope: $\beta$ = +3.2 m$^3$/s/year, CI = [2.1, 4.3]
    \item Pettitt: $K_\tau$ = 486, $\tau$ = 32 (year 2002), p = 0.0001
    \item Mean before 2002: 125 m$^3$/s; Mean after 2002: 178 m$^3$/s
    \item Urbanization project completed in 2002
\end{itemize}

\vspace{0.3cm}

\textbf{(a) What type(s) of non-stationarity are present?} \points{4}

\answer{Solution:}

\textbf{BOTH} types of non-stationarity are present:

\begin{enumerate}
    \item \textbf{Trend:} Mann-Kendall p = 0.0003 $\ll$ 0.05
    \begin{itemize}
        \item Highly significant increasing trend
        \item Sen's slope = +3.2 m$^3$/s/year confirms upward change
    \end{itemize}

    \item \textbf{Change Point:} Pettitt p = 0.0001 $\ll$ 0.05
    \begin{itemize}
        \item Highly significant abrupt shift in 2002
        \item Mean jumped from 125 to 178 m$^3$/s (increase of 53 m$^3$/s = 42\%)
    \end{itemize}
\end{enumerate}

\explanation{
This is a complex case where urbanization caused BOTH:
\begin{itemize}
    \item An immediate jump in flow (change point) due to increased impervious surfaces
    \item A continuing upward trend as development continued
\end{itemize}
The physical evidence (2002 urbanization) validates both statistical findings. When change points and trends coexist, the Mann-Kendall test may detect a "trend" that's actually driven by the step change.
}

\vspace{0.3cm}

\textbf{(b) Which dataset should you use for bridge design (75-year life)?} \points{4}

\answer{Solution:}

\textbf{Use post-2002 data only} (last 22 years of the 50-year record):

\begin{itemize}
    \item Represents \textbf{current urbanized conditions}
    \item Mean = 178 m$^3$/s (NOT 125 m$^3$/s from pre-urban period)
    \item Account for continuing trend: Project forward conservatively
    \item Example projection for Year 25 after change:
    $$178 + (3.2 \times 25) = 178 + 80 = 258 \text{ m}^3\text{/s}$$
\end{itemize}

\explanation{
\textbf{Why NOT use all 50 years?}
\begin{itemize}
    \item Pre-2002 data represents a different (non-urbanized) watershed
    \item Mixing regimes gives incorrect frequency estimates
    \item Design would be based on outdated conditions
\end{itemize}

\textbf{Why NOT use pre-2002 data?}
\begin{itemize}
    \item No longer represents reality - urbanization changed the watershed
    \item Would severely underestimate design flows (125 vs 178 m$^3$/s)
\end{itemize}

The bridge will exist in the post-urbanization reality, so use that data.
}

\vspace{0.3cm}

\textbf{(c) Additional considerations and safety factors?} \points{4}

\answer{Solution:}

\begin{enumerate}
    \item \textbf{Account for continuing trend:}
    \begin{itemize}
        \item Project mean flow for mid-design life (Year 37.5)
        \item Add uncertainty from Sen's slope CI: use upper bound (4.3 m$^3$/s/year)
    \end{itemize}

    \item \textbf{Limited post-change record:}
    \begin{itemize}
        \item Only 22 years of urbanized data - consider regional pooling
        \item Increase safety factor by 10-20\% for uncertainty
    \end{itemize}

    \item \textbf{Climate change considerations:}
    \begin{itemize}
        \item 75-year bridge life extends to 2100+
        \item Consider climate projections for region
        \item Non-stationary frequency analysis methods
    \end{itemize}

    \item \textbf{Adaptive management:}
    \begin{itemize}
        \item Design for future upgrades if possible
        \item Plan monitoring program
        \item Re-evaluate every 10-15 years
    \end{itemize}
\end{enumerate}

\explanation{
Long-lived infrastructure in non-stationary environments requires:
\begin{itemize}
    \item Conservative design based on projected future, not historical average
    \item Explicit consideration of uncertainty through safety factors
    \item Flexibility for future adaptation
    \item Documentation of assumptions and methods
\end{itemize}
The worst outcome would be a bridge that fails before its design life due to underestimated flows.
}

\newpage

\subsection*{Question 2 \points{13}}

\textbf{Given Results:}
\begin{center}
\begin{tabular}{|l|c|c|}
\hline
Test & Result & P-value \\
\hline
Mann-Kendall S & -245 & 0.12 \\
Sen's Slope & -1.8 mm/year & -- \\
Pettitt $K_\tau$ & 98 & 0.08 \\
\hline
\end{tabular}
\end{center}

Authority wants to immediately reduce water supply by 30\%.

\vspace{0.3cm}

\textbf{(a) Are results statistically significant at $\alpha = 0.05$?} \points{3}

\answer{Solution:}

\textbf{NO, neither result is statistically significant:}

\begin{itemize}
    \item Mann-Kendall: p = 0.12 $>$ 0.05 $\Rightarrow$ \textbf{Not significant}
    \item Pettitt: p = 0.08 $>$ 0.05 $\Rightarrow$ \textbf{Not significant}
\end{itemize}

At $\alpha = 0.05$, we \textbf{fail to reject $H_0$} for both tests.

\explanation{
The observed patterns (negative S, some evidence of change point) could reasonably occur by chance. While p = 0.08 and 0.12 suggest some evidence, it's not strong enough to meet the standard 5\% significance threshold. These are "suggestive but inconclusive" results.
}

\vspace{0.3cm}

\textbf{(b) Should they make major operational changes based solely on these results?} \points{5}

\answer{Solution:}

\textbf{NO - Major decision not justified by current evidence:}

\begin{enumerate}
    \item \textbf{Statistical evidence is weak:}
    \begin{itemize}
        \item Neither trend nor change point is significant at $\alpha = 0.05$
        \item Observed patterns could be due to natural variability
        \item 30\% reduction is a drastic action for inconclusive evidence
    \end{itemize}

    \item \textbf{Economic and social consequences:}
    \begin{itemize}
        \item Reducing supply by 30\% has major impacts on users
        \item Could cause water rationing, affect industries
        \item Decision should be based on stronger evidence
    \end{itemize}

    \item \textbf{Better approach:}
    \begin{itemize}
        \item Implement water conservation measures (smaller, reversible)
        \item Continue monitoring and data collection
        \item Plan for contingencies but don't commit to drastic cuts
    \end{itemize}
\end{enumerate}

\explanation{
\textbf{Engineering judgment principle:} The magnitude of the response should match the strength of evidence. Weak statistical evidence (p $>$ 0.05) does not justify major irreversible decisions. However, the concerning pattern (negative trend, p = 0.12) does warrant attention and precautionary measures.

\textbf{Risk assessment:}
\begin{itemize}
    \item Type I error (false alarm): Cut supply when not needed $\Rightarrow$ economic damage
    \item Type II error (missed threat): Don't act when needed $\Rightarrow$ water shortage
\end{itemize}
Balance these risks with proportional responses.
}

\vspace{0.3cm}

\textbf{(c) What additional analyses would you recommend? (At least 3)} \points{5}

\answer{Solution:}

\begin{enumerate}
    \item \textbf{Physical investigation:}
    \begin{itemize}
        \item Identify any known watershed changes (land use, climate patterns)
        \item Check for data quality issues (gauge problems, measurement changes)
        \item Look for physical explanation of apparent decline
    \end{itemize}

    \item \textbf{Regional analysis:}
    \begin{itemize}
        \item Compare with neighboring watersheds/reservoirs
        \item If regional pattern exists $\Rightarrow$ more likely real
        \item If only local $\Rightarrow$ may be site-specific issue
    \end{itemize}

    \item \textbf{Extended dataset:}
    \begin{itemize}
        \item Collect 5-10 more years of data
        \item Re-run tests with larger sample size
        \item More data increases statistical power
    \end{itemize}

    \item \textbf{Alternative statistical methods:}
    \begin{itemize}
        \item Spearman's rho test for trends
        \item CUSUM analysis for change points
        \item Bootstrap confidence intervals for uncertainty
    \end{itemize}

    \item \textbf{Climate analysis:}
    \begin{itemize}
        \item Check correlation with climate indices (PDO, AMO, ENSO)
        \item Analyze precipitation trends in watershed
        \item Consider multi-decadal natural cycles
    \end{itemize}

    \item \textbf{Water balance modeling:}
    \begin{itemize}
        \item Model inflows under different scenarios
        \item Estimate future supply under various conditions
        \item Quantify risk of shortage with confidence intervals
    \end{itemize}
\end{enumerate}

\explanation{
\textbf{Key principle:} Before making major decisions with weak statistical evidence, gather more information:
\begin{itemize}
    \item \textbf{More data} $\rightarrow$ stronger statistical power
    \item \textbf{Multiple lines of evidence} $\rightarrow$ more confidence
    \item \textbf{Physical understanding} $\rightarrow$ validates or questions statistics
\end{itemize}

If the decline is real, these additional analyses will strengthen the case. If it's natural variability, the extra investigation will prevent costly mistakes. Either way, the authority makes a better-informed decision.
}

\vspace{1cm}

\begin{center}
\fbox{\parbox{0.9\textwidth}{
\centering
\Large\textbf{END OF SOLUTIONS}\\[0.3cm]
\normalsize
\textit{Key Takeaways:}
\begin{itemize}
    \item Statistical significance requires p-value $<$ $\alpha$ (typically 0.05)
    \item Mann-Kendall detects gradual trends; Pettitt detects abrupt changes
    \item Sen's slope quantifies the magnitude of change
    \item Engineering decisions should combine statistical evidence with physical understanding
    \item The strength of response should match the strength of evidence
\end{itemize}
}}
\end{center}

\end{document}
