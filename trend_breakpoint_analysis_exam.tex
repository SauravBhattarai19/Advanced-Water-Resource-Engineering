\documentclass[11pt,letterpaper]{article}
\usepackage[margin=1in]{geometry}
\usepackage{amsmath,amssymb}
\usepackage{graphicx}
\usepackage{enumitem}
\usepackage{booktabs}
\usepackage{array}
\usepackage{fancyhdr}
\usepackage{lastpage}
\usepackage{tikz}
\usepackage{xcolor}

% Header and Footer
\pagestyle{fancy}
\fancyhf{}
\lhead{CEE 5XX: Advanced Water Resources Engineering}
\rhead{Trend \& Breakpoint Analysis Exam}
\cfoot{Page \thepage\ of \pageref{LastPage}}

% Custom commands
\newcommand{\points}[1]{\textbf{[#1 points]}}
\newcommand{\blank}[1]{\underline{\hspace{#1}}}

\begin{document}

% Title Section
\begin{center}
    \Large\textbf{Trend Detection and Breakpoint Analysis Exam}\\[0.3cm]
    \large CEE 5XX: Advanced Water Resources Engineering\\
    Jackson State University\\[0.3cm]
    \normalsize
    \textbf{Duration:} 1 Hour \quad | \quad \textbf{Total Points:} 100
\end{center}

\vspace{0.3cm}

\noindent\textbf{Name:} \blank{8cm} \hfill \textbf{Date:} \blank{3cm}

\vspace{0.3cm}

\noindent\fbox{\parbox{\textwidth-2\fboxsep}{
\textbf{Instructions:}
\begin{itemize}[noitemsep,topsep=2pt]
    \item Answer ALL questions
    \item Show all calculations for numerical problems
    \item Calculator and formula sheet are permitted
    \item No electronic devices except calculator
\end{itemize}
}}

\vspace{0.5cm}

%==============================================================================
\section*{Part A: Multiple Choice Questions \points{30}}
%==============================================================================

\textit{Circle the best answer for each question. Each question is worth 5 points.}

\vspace{0.3cm}

\begin{enumerate}[leftmargin=*]

\item A hydrologic time series dataset has a strongly skewed distribution with several extreme outliers. Which test is most appropriate for detecting monotonic trends?

\begin{enumerate}[label=(\alph*)]
    \item Linear regression with t-test
    \item Mann-Kendall test
    \item Two-sample t-test
    \item Pearson correlation test
\end{enumerate}

\vspace{0.5cm}

\item In the Mann-Kendall test, if the S statistic equals +180 for a dataset with 25 observations, and the calculated p-value is 0.08, what should you conclude at $\alpha = 0.05$?

\begin{enumerate}[label=(\alph*)]
    \item Reject $H_0$; significant increasing trend exists
    \item Fail to reject $H_0$; no significant trend detected
    \item The trend is decreasing because S is positive
    \item The test is invalid due to small sample size
\end{enumerate}

\vspace{0.5cm}

\item Which statement correctly describes the difference between trends and change points in hydrologic data?

\begin{enumerate}[label=(\alph*)]
    \item Trends are detected by Pettitt test; change points by Mann-Kendall
    \item Trends are gradual monotonic changes; change points are abrupt shifts
    \item Both represent the same type of non-stationarity
    \item Change points occur only in spatial data
\end{enumerate}

\vspace{0.5cm}

\item You calculate Sen's slope = +2.5 mm/year with 95\% CI = [+1.8, +3.2] for annual rainfall. The Mann-Kendall p-value is 0.02. What does this indicate?

\begin{enumerate}[label=(\alph*)]
    \item No significant trend because the slope is small
    \item Significant increasing trend with high confidence in magnitude
    \item Decreasing trend since confidence interval is positive
    \item Results are inconclusive without the mean rainfall value
\end{enumerate}

\vspace{0.5cm}

\item For a Pettitt test, what does the $U_{t,T}$ statistic represent?

\begin{enumerate}[label=(\alph*)]
    \item The mean difference between two segments
    \item The sum of pairwise rank comparisons between segments before and after time $t$
    \item The variance ratio between two segments
    \item The correlation coefficient between time and data values
\end{enumerate}

\vspace{0.5cm}

\item A stream discharge dataset (40 years) shows: Mann-Kendall p = 0.03, Sen's slope = +2.1 m$^3$/s/year. The post-change period after a detected dam construction has only 3 years of data. What is the most appropriate engineering action?

\begin{enumerate}[label=(\alph*)]
    \item Use the 3-year post-change period for design
    \item Ignore the change point and use all 40 years
    \item Document the change point but use regional pooling or conservative design factors
    \item Wait indefinitely until post-change data reaches 30 years
\end{enumerate}

\end{enumerate}

\newpage

%==============================================================================
\section*{Part B: Short Numerical Problems \points{45}}
%==============================================================================

\subsection*{Problem 1: Mann-Kendall S Statistic Calculation \points{15}}

You are given the following \textbf{partial} Mann-Kendall analysis for 8 years of annual rainfall data (mm):

\begin{center}
\begin{tabular}{|c|c|c|c|c|c|c|c|c|}
\hline
Year & 1 & 2 & 3 & 4 & 5 & 6 & 7 & 8 \\
\hline
Rainfall (mm) & 850 & 870 & 865 & 890 & 900 & 895 & 920 & 930 \\
\hline
\end{tabular}
\end{center}

\vspace{0.3cm}

The following pairwise comparisons have been completed:

\begin{itemize}[noitemsep]
    \item Comparisons involving Year 1: All 7 comparisons yield $sgn = +1$ (S contribution = +7)
    \item Comparisons involving Year 2: 5 positive, 1 negative (S contribution = +4)
    \item Comparisons involving Year 3: All 5 comparisons yield $sgn = +1$ (S contribution = +5)
    \item Comparisons involving Year 4: 3 positive, 1 negative (S contribution = +2)
    \item Comparisons involving Year 5: 2 positive, 1 negative (S contribution = +1)
    \item Comparisons involving Year 6: All 2 comparisons yield $sgn = +1$ (S contribution = +2)
    \item Comparisons involving Year 7: 1 positive (S contribution = +1)
\end{itemize}

\textbf{Tasks:}

\begin{enumerate}[label=(\alph*)]
    \item Calculate the total Mann-Kendall S statistic. \points{3}

    \vspace{2cm}

    \item Calculate the variance of S using: $\text{Var}(S) = \frac{n(n-1)(2n+5)}{18}$ \points{4}

    \vspace{3cm}

    \item Calculate the standardized Z-score. (Use $Z = \frac{S-1}{\sqrt{\text{Var}(S)}}$ since $S > 0$) \points{4}

    \vspace{3cm}

    \item What is your conclusion about the trend? (Hint: For $|Z| > 1.96$, p < 0.05) \points{4}

    \vspace{2cm}
\end{enumerate}

\newpage

\subsection*{Problem 2: Sen's Slope Estimator \points{15}}

For a 5-year dataset of annual peak discharge (m$^3$/s):

\begin{center}
\begin{tabular}{|c|c|c|c|c|c|}
\hline
Year & 1 & 2 & 3 & 4 & 5 \\
\hline
Discharge & 100 & 108 & 110 & 118 & 125 \\
\hline
\end{tabular}
\end{center}

\vspace{0.3cm}

The following pairwise slopes $Q_{ij} = \frac{X_j - X_i}{j - i}$ have been calculated:

\begin{center}
\begin{tabular}{|c|c|c|c|c|}
\hline
Pair & Calculation & Slope ($Q_{ij}$) & Pair & Slope ($Q_{ij}$) \\
\hline
$Q_{12}$ & $(108-100)/(2-1)$ & 8.00 & $Q_{25}$ & 5.67 \\
$Q_{13}$ & $(110-100)/(3-1)$ & 5.00 & $Q_{34}$ & 8.00 \\
$Q_{14}$ & $(118-100)/(4-1)$ & 6.00 & $Q_{35}$ & 7.50 \\
$Q_{15}$ & $(125-100)/(5-1)$ & 6.25 & $Q_{45}$ & 7.00 \\
$Q_{23}$ & $(110-108)/(3-2)$ & 2.00 & & \\
$Q_{24}$ & $(118-108)/(4-2)$ & 5.00 & & \\
\hline
\end{tabular}
\end{center}

\textbf{Tasks:}

\begin{enumerate}[label=(\alph*)]
    \item Sort all 10 slopes in ascending order. \points{3}

    \vspace{2cm}

    \item Calculate the Sen's slope estimator $\beta$ (median of all slopes). \points{4}

    \vspace{3cm}

    \item Interpret the meaning of this slope in engineering terms. \points{4}

    \vspace{3cm}

    \item If this trend continues, what would be the projected discharge in Year 10? \points{4}

    \vspace{3cm}
\end{enumerate}

\newpage

\subsection*{Problem 3: Pettitt Test for Change Point Detection \points{15}}

A hydrologist analyzed 12 years of annual peak flow data. The $U_{t,T}$ statistics for each potential split point $t$ have been calculated as follows:

\begin{center}
\begin{tabular}{|c|c|c||c|c|c|}
\hline
$t$ & $U_{t,T}$ & $|U_{t,T}|$ & $t$ & $U_{t,T}$ & $|U_{t,T}|$ \\
\hline
1 & -8 & 8 & 7 & -42 & 42 \\
2 & -18 & 18 & 8 & -36 & 36 \\
3 & -26 & 26 & 9 & -28 & 28 \\
4 & -32 & 32 & 10 & -18 & 18 \\
5 & -38 & 38 & 11 & -6 & 6 \\
6 & -44 & 44 & & & \\
\hline
\end{tabular}
\end{center}

\textbf{Tasks:}

\begin{enumerate}[label=(\alph*)]
    \item Determine the test statistic $K_\tau = \max_{1 \le t < T} |U_{t,T}|$ \points{3}

    \vspace{2cm}

    \item Identify the change point location $\tau$ (the value of $t$ where $|U_{t,T}|$ is maximum) \points{3}

    \vspace{2cm}

    \item Calculate the p-value using: $p \approx 2 \exp\left(-\frac{6K_\tau^2}{T^3 + T^2}\right)$ where $T = 12$ \points{5}

    \vspace{4cm}

    \item Is there a significant change point at $\alpha = 0.05$? What does the negative $U_\tau$ indicate about the direction of change? \points{4}

    \vspace{3cm}
\end{enumerate}

\newpage

%==============================================================================
\section*{Part C: Analytical Interpretation Questions \points{25}}
%==============================================================================

\subsection*{Question 1 \points{12}}

You are a consulting engineer analyzing 50 years of streamflow data for a watershed. Your analysis yields the following results:

\begin{itemize}[noitemsep]
    \item \textbf{Mann-Kendall Test:} S = +892, p-value = 0.0003
    \item \textbf{Sen's Slope:} $\beta$ = +3.2 m$^3$/s/year, 95\% CI = [2.1, 4.3]
    \item \textbf{Pettitt Test:} $K_\tau$ = 486, $\tau$ = 32 (year 2002), p-value = 0.0001
    \item \textbf{Mean flow before 2002:} 125 m$^3$/s
    \item \textbf{Mean flow after 2002:} 178 m$^3$/s
    \item \textbf{Known event:} Major urbanization project completed in 2002
\end{itemize}

\vspace{0.3cm}

\textbf{Based on these results:}

\begin{enumerate}[label=(\alph*)]
    \item What type(s) of non-stationarity are present in the data? Explain your reasoning. \points{4}

    \vspace{4cm}

    \item If you were designing a new bridge with a 75-year design life, which dataset should you use for frequency analysis and why? \points{4}

    \vspace{4cm}

    \item What additional considerations or safety factors would you recommend? \points{4}

    \vspace{4cm}
\end{enumerate}

\newpage

\subsection*{Question 2 \points{13}}

A municipal water authority presents you with the following analysis of their reservoir inflow data (35 years):

\begin{center}
\begin{tabular}{|l|c|c|}
\hline
\textbf{Statistical Test} & \textbf{Result} & \textbf{P-value} \\
\hline
Mann-Kendall S statistic & -245 & 0.12 \\
Sen's Slope & -1.8 mm/year & -- \\
Pettitt Test $K_\tau$ & 98 & 0.08 \\
\hline
\end{tabular}
\end{center}

The authority is concerned about declining water availability and wants to immediately reduce their water supply commitments by 30\% based on these results.

\vspace{0.3cm}

\textbf{Provide your professional engineering assessment:}

\begin{enumerate}[label=(\alph*)]
    \item Are the detected trend and change point statistically significant at $\alpha = 0.05$? \points{3}

    \vspace{3cm}

    \item Should the authority make major operational changes based solely on these results? Justify your answer. \points{5}

    \vspace{5cm}

    \item What additional analyses or data would you recommend before making such decisions? List at least 3 specific recommendations. \points{5}

    \vspace{6cm}
\end{enumerate}

\vspace{1cm}

\begin{center}
\fbox{\parbox{0.8\textwidth}{
\centering
\textbf{END OF EXAM}\\[0.2cm]
\textit{Please review your answers before submitting.}\\
\textit{Remember: Good engineering judgment combines statistical evidence with physical understanding.}
}}
\end{center}

\end{document}
